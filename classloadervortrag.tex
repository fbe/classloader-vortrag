\documentclass{beamer}
\usepackage{ngerman}
\usepackage[utf8]{inputenc} 

\usetheme{Warsaw}  %% Themenwahl
 
\title{Java Classloading für Fortgeschrittene}
\author{Felix Becker}
\date{\today}
 
\begin{document}
\maketitle
\frame{\tableofcontents}
 
\section{Classloading}
	\begin{frame}
		\frametitle{Classloader in der JVM}
		\begin{itemize}
			\item{BootstrapClassLoader}

			\begin{itemize}
				\item{lädt JVM Basisklassen aus jre/lib/}
				\item{nativ implementiert}
				\item{java.lang.ClassLoader private native Class findBootstrapClass}
			\end{itemize}

			\pause

			\item{ExtensionsClassLoader}
			\begin{itemize}
				\item{lädt Extension-Klassen aus jre/lib/ext/ bzw. java.ext.dirs}
				\item{com.sun.misc.Launcher\$ExtClassLoader}
			\end{itemize}

			\pause

			\item{SystemClassLoader}
			\begin{itemize}
				\item{java.class.path (com.sun.misc.Launcher\$AppClassLoader)}
			\end{itemize}
		\end{itemize}
	\end{frame}

	\begin{frame}
		\frametitle{ServletStandard}
		\begin{itemize}
			\item{Eigener Classloader für jede Applikation}
			\item{Container bringt ebenfalls eigene Classloader mit (bsp. tomcat/lib/)}
			\item{WebAppClassloader sucht zuerst in der Webapp und fragt erst danach die Parent-Classloader}
			\item{Library-Entwickler: Thread.currentThread.getContextClassLoader verwenden!}
		\end{itemize}
	\end{frame}

\section{Java Bytecode}
	\begin{frame}
		\frametitle{Aufbau einer Java-Klasse}
		\begin{itemize}
			\item{Hier entsteht für Sie ein schönes Bild}
		\end{itemize}
	\end{frame}
	% details ueber frames, operand stacks, local variables etc.

\section{Java Instrumentation}
	\begin{frame}
		\frametitle{Java Agent / Instrumentation API}	
		\begin{itemize}
			\item{Mehrere Möglichkeiten in die JVM zu kommen:}
			\begin{itemize}
				\item{JVM TI (Java Virtual Machine Tooling Interface - JSR 163 - JavaTM Platform Profiling Architecture)}
				\begin{itemize}
					\item{Seit Java 5}
					\item{C / C++ - Wird als shared object in die JVM geladen}
					\item{Hohe Performance aber auch hohe Crashgefahr}
				\end{itemize}

				\item{Java-Agent}

				\begin{itemize}
					\item{java.lang.instrument}
					\item{Class transformer definition}
					\item{Hook in den native call defineClass}
					\item{Mehrere Agents in der selben JVM erlaubt}
					\item{Agents theoretisch dynamisch zur Runtime nachladbar}
				\end{itemize}
			\end{itemize}
		\end{itemize}
	\end{frame}

\end{document}

